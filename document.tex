

\newpage
\section*{Úloha 4: Obvod Rocheova laloku v ekvatoriální rovině}

\textit{(numerické řešení rovnic, analýza chyby, extrapolace)}

V popisu dynamiky dvojhvězdných systémů je definován Rocheův lalok jako oblast prostoru kolem hvězdy 1, kterou lze zaplnit plynem, bez toho, aby přetékal na hvězdu 2. Tato oblast je ohraničena ekvipotenciální plochou v systému korotujícím se vzájemně se obíhajícími hvězdami, na níž je potenciál roven potenciálu v Lagrangeově bodě $L_1$ (bod kde se vyrovnávají gravitační síly obou těles a odstředivá síla). Pro tuto úlohu se omezíme na rovinu v níž obíhají obě hvězdy po kruhové dráze. V této rovině nalezněte čáru ohraničující tento lalok a spočtěte její délku. Počátek souřadného systému položte do hvězdy 1 a osa $x$ nechť směřuje do hvězdy 2. Délky budeme měřit v jednotkách daných vzdáleností obou hvězd $R$. Podrobnější instrukce:

\begin{itemize}
\item Najděte polohu bodu $L_1$ (viz rovnice vedle obrázku) a hodnotu potenciálu $V_L$ v tomto bodě.
\item Pro hodnoty $\theta = 2\pi n / N$, $n = 0,1,\dots,N$ nalezněte body $x_n = r\cos \theta$, $y_n = r\sin \theta$ na hledané ekvipotenciální ploše. Body hledejte pro fixní $\theta$ řešením rovnice $V(r) = V_L$ vámi zvolenou metodou. Pokuste se najít hodnoty $x,y$ s přesností blízkou strojovému $\epsilon$.
\item Spočtěte délku $D_N$ lomené čáry $(x_1, y_1), (x_2, y_2), \ldots, (x_N, y_N)$. Studujte konvergenci této hodnoty pro $N \to \infty$. Pokuste se identifikovat bod, kde dochází k rovnováze mezi zaokrouhlovací a diskretizační chybou.
\item Pokuste se přesnost metody zlepšit například vhodnou extrapolací diskretizační chyby.
\end{itemize}

Výstupem k odevzdání bude obrázek lomené čáry $(x_1, y_1), (x_2, y_2), \ldots, (x_N, y_N)$, dále obrázek dokumentující konvergenci $D_N$, z nějž bude vidět řád diskretizační chyby a nakonec co nejpřesnější hodnota $D = \lim_{N \to \infty} D_N$.

\begin{center}
\includegraphics[width=0.5\textwidth]{img-0.jpeg}
\end{center}

Rovnice pro určení souřadnice $x_L$ Lagrangeova bodu $L_1$ je dána rovnováhou sil:
\[
x_L - m_2 + \frac{m_2}{(1 - x_L)^2} - \frac{m_1}{x_L^2} = 0.
\]
Kde $m_1 = M_1 / M$ a $m_2 = M_2 / M$ jsou hmotnosti obou složek v jednotkách celkové hmotnosti $M = M_1 + M_2$ dvojhvězdy. Gravitační potenciál (v jednotkách $\kappa M / R$) v korotujícím systému (včetně odstředivé síly) je
\[
V(x, y) = -\frac{m_1}{r} - \frac{m_2}{r_2} - \frac{1}{2} r_0^2,
\]
kde $r = \sqrt{x^2 + y^2}$, $r_2 = \sqrt{r^2 - 2r\cos\theta + 1}$ a $r_0 = \sqrt{r^2 - 2m_2r\cos\theta + m_2^2}$ jsou vzdálenosti bodu $(x,y)$ od hvězdy 1, hvězdy 2 a od těžiště dvojhvězdného systému (tj. středu rotace).

\newpage

\section*{Úloha 5: Model oscilující chemické reakce}

\textit{(řešení obyčejných diferenciálních rovnic)}

V dnešní době je známo několik tzv. chemických oscilátorů (doporučuji vyhledat „chemical oscillator“ na youtube), což směs látek, v níž dochází k jisté množině chemických reakcí, které vedou k oscilacím koncentrace meziproduktů. V době svého objevu to vedlo k nedůvěře a odmítnutí (viz heslo \textit{Belousov-Zhabotinsky reaction} na anglické wikipedii), ale dnes existují jednoduché teoretické modely, které takové chování vysvětlují. Jeden takový jednoduchý model si zkusíme prozkoumat. V našem modelu (Lotka-Volterra model) se látka A mění postupně na látku B, přičemž vznikají meziprodukty X a Y podle schématu
\[
\mathrm{A} + \mathrm{X} \rightarrow 2\mathrm{X}
\]
\[
\mathrm{X} + \mathrm{Y} \rightarrow 2\mathrm{Y}
\]
\[
\mathrm{Y} \rightarrow \mathrm{B}
\]
Označme-li $k_{1}$, $k_{2}$ a $k_{3}$ reakční rychlosti těchto reakcí, dostaneme pro časové závislosti koncentrace $A$, $X$, $Y$ a $B$ soustavu diferenciálních rovnic
\[
\begin{array}{l}
\frac{d}{dt} A = -k_{1}AX, \quad
\frac{d}{d\tau} a = -\kappa ax, \\
\frac{d}{dt} X = k_{1}AX - k_{2}XY, \quad
\frac{d}{d\tau} x = \kappa ax - xy, \\
\frac{d}{dt} Y = k_{2}XY - k_{3}Y, \quad
\frac{d}{d\tau} y = xy - y, \\
\frac{d}{dt} B = k_{3}Y, \quad
\frac{d}{d\tau} b = y.
\end{array}
\]
V pravém sloupci jsme rovnice přepsali do bezrozměrných veličin $\tau = tk_{3}$, $b = B / A_{0}$, $y = Y / A_{0}$, $x = k_{2}X / k_{3}$ a $\kappa = k_{1} / k_{2}$, přičemž $A_{0}$ je počáteční koncentrace látky A. Řešte tyto rovnice numericky pro $\kappa = 0.002$ a počáteční podmínku $a = 498.4$, $x = 1.5$, $y = 0.1$, $b = 0$. Podrobněji:
\begin{itemize}
\item Napište proceduru pro integraci dané soustavy diferenciálních rovnic s řádem přesnosti 4. řádu. Ověřte konvergenci výsledků v závislosti na délce kroku pro výsledek ve fixním čase $t = 100$ a dále ověřte, že veličina $a + x + b + y$ se zachovává.
\item Nakreslete $a(t)$ a $b(t)$ do jednoho obrázku a naleznete hodnotu času $t$, kdy se vyrovná koncentrace reaktantů a produktů $a(t) = b(t)$. Odhadněte jak přesně jste tento čas určili.
\item Nakreslete koncentrace meziproduktů $x(t)$ a $y(t)$ a určete periodu jejich oscilací. Odhadněte jaká je přibližně vaše přesnost určení této periody?
\end{itemize}

Rozmyslete si, že oscilace fungují jako jakási „chemická pumpa“. Část cyklu s vysokým $x$ vede ke zvýšené konzumaci reaktantu A a část cyklu s vysokým $y$ vede ke zvýšené tvorbě produktu B.

Výstupem budou tři obrázky (rychlost konvergence, časová závislost $a(t)$, $b(t)$ a časová závislost meziproduktů) a dvě čísla (čas vyrovnání koncentrací a perioda oscilací).

\end{document}